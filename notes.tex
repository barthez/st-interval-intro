\documentclass[11pt,a4paper]{report}
\usepackage[utf8]{inputenc}
\usepackage[T1]{fontenc}
\usepackage[polish]{babel}
\usepackage{graphicx}
\usepackage{lmodern}
\usepackage{url}
\usepackage{hyperref}

\begin{document}
\begin{center}
  Analiza odcina ST
\end{center}

\textbf{1. Metody znajdowania odcinka ST}

 - Pomiędzy (pkt. J) $t_R+40ms$, a (pkt. ST) $t_R+60ms$, tylko w zespołach SV

 - Dynamiczne $t_R+40ms+0,948\sqrt{t_{RR}}$, a $t_R+40ms+2,21\sqrt{t_{RR}}$

 Parametry algorytmu \emph{(statyczne/dynamiczne wyznaczanie ST = statyczne, minimalna
 długość epizodu = 60ms, minimalny odstęp miedzy epizodami = 30ms)}

 Dane wejściowe: WAVES (położenie i typ zespołu QRS)

\textbf{Analiza odcinka ST}

  - Nachylenie odcinka: tangens konta prostej przechodzącej przez punkty J i
  ST

\textbf{Prezentacja danych}

  Wykres prezentujący tangens nachylenia odcinka ST, gdzie cecha: "poziomy",
  "narastający", "opadający" będzie oznaczona odpowiednim kolorem (jeśli
  wartość przekroczy podane ograniczenie na odcinek stały).

  Wykres prezentujący obniżenie odcinka ST w mm ($1mV = 10mm$).

  Oba powyższe wykresy będą próbkowane niejednorodnie.

  Dodatkowo zaznaczanie punktów J (początek odcinka ST) i punktu ST. Oraz
  Rysowanie odcinka ST na wykresie sygnału EKG.

  Epizody ST: Lista episodów wraz z parametrami (moment wystąpienia, czas,
  maksymalne wartości odcinka ST [max nachylenie, max obniżenie/podniesienie],
  rytm serca przed, po i w trakcie epizodu). Wybranie epizodu pokazuje go na
  wykresie sygnału EKG.

\end{document}
