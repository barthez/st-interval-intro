\documentclass{beamer}
\usepackage[utf8]{inputenc}
\usepackage[T1]{fontenc}
\usepackage[polish]{babel}
\usepackage{graphicx}
\usepackage{lmodern}
\usepackage{url}
\usepackage{hyperref}

\usetheme{AGH}

\title[ST Interval]{Analiza odcinka ST}

\author[B. Bułat, K. Piekutowski]{Bartłomiej Bułat, Krzysztof Piekutowski}

\date[2012]{19.12.2012}

\institute[AGH]
{Wydział EAIiIB\\ 
Katedra Automatyki
}

\setbeamertemplate{itemize item}{$\maltese$}

\begin{document}

{
\usebackgroundtemplate{\includegraphics[width=\paperwidth]{titlepagepl}} % wersja polska
 \begin{frame}
   \titlepage
 \end{frame}
}


\begin{frame}
  \frametitle{Analiza odcina ST - po co?}

  \begin{itemize}
    \item Choroba niedokrwienna serca (Zawał)
    \item Miażdżyca
    \item Choroba wieńcowa
  \end{itemize}

\end{frame}

\begin{frame}
  \frametitle{Założenia modułu}
  \begin{itemize}
    \item Wykryte punkty EKG: $ORS_{onset}$ ($ISO$), $ORS_{end}$
      ($J$), $T_{end}$
    \item Zapis EKG z usuniętym falowaniem linii izoelektrycznej
    \item Sklasyfikowane zespoły QRS (Analizowane są tylko zespoły
      przedkomorowe)
    \item Obliczone i przepróbkowane HR
  \end{itemize}
\end{frame}

\begin{frame}
  \frametitle{Algorytm, krok 1: Wykrycie punktu $T_{peak}$}

  Punkt $T_{peak}$ znajduje się pomiędzy punktem $J$, punktem $T_{end}$.

  

  Wyznaczenie punktu polega na przeprowadzeniu 5-warstwowej dekompozycji
  falkowej splinów 4-stopnia, wyliczeniu punktu przejścia przez zero różnicy
  pierwszego rzędu pary max-min. Ten punkt przejścia przez zero jest punktem
  $T_{peak}$.

\end{frame}

\begin{frame}
  \frametitle{Algorytm, krok 2: Wykrycie punkty $T_{onset}$ ($ST_{end}$)}

  Punkt $T_{onset}$ znajduje się pomiędzy punktem $J20$ (znajdującym się $20ms$
  za punktem $J$), a punktem $T_{peak}$.

  Aby znaleźć ten punkt należy wyznaczyć prostą łączącą punkty $J20$ i
  $T_{peak}$. Punkt $P_{onset}$ znajduje się w miejscu największej różnicy
  miedzy wyznaczoną prostą, a sygnałem.

\end{frame}

\begin{frame}
  \frametitle{Algorytm, krok 3: przesunięcie odcinka ST}

  Przesunięcie nie jest mierzone żadną wartością, a jedynie określane, jako
  \emph{Wyższe}, \emph{Niższe} lub \emph{Normalne}.

  Przesunięcie odcinka ST (wg. algorytmu $J+X$) to różnica między wartością
  sygnału w punkcie $JX$, a wartością w punkcje $ISO$:

  \begin{center}
  \begin{math}
    offset = y(J+X) - y(ISO)
  \end{math}
  \end{center}

  Gdzie $X$ należy do przedziału $60ms$ do $80ms$ i jest ściśle uzależnione 
  od aktualnej wartości $HR$. Im większe $HR$ tym mniejsza wartość $X$.
\end{frame}

\begin{frame}
  \frametitle{Algorytm, krok 4: typ odcinka ST}

  Aby wskazać czy odcinek ST jest prosty czy zakrzywiony należy wyznaczyć
  prostą między punktem $J20$ a punktem $TE$. Punkt $TE$ to w zależności od
  przesunięcia odcinka ST, punkt $T_{peak}$ jeśli przesunięcie jest
  \emph{Wyższe}, lub punkt $T_{onset}$ jeśli przesunięcie jest \emph{Normalne}
  lub \emph{Niższe}.

  Następnie szukana jest największa odległość między wyznaczoną prostą, a
  sygnałem. Jeżeli ta odległość jest większa od założonego progu, to odcinek ST
  jest zakrzywiony, jeśli zaś mniejsza to odcinek jest prosty.
\end{frame}

\begin{frame}
  \frametitle{Algorytm, krok 5.1: zakrzywienie odcinka ST}

  Aby określić kierunek wykrzywienia odcinka ST, należy policzyć stosunek
  liczny punktów sygnały na odcinku $J20$ do $TE$, które znajdują się pod i nad
  prostą łączącą te punkty do ilości wszystkich punktów na tym odcinku.

  Jeżeli stosunek ilości punktów nad prostą do ilości wszystkich punktów jest
  większy od $0.7$ odcinek ST się wznosi, jeżeli ponad $70\%$ punktów znajduje
  się poniżej, odcinek ST opada. 
  
  Jeżeli żaden z tych wyznaczników nie przekroczył wartości $0.7$ należy
  przybliżyć odcinek ST za pomocą krzywej (np. paraboli), ponieważ krzywa jest
  zbyt zaszumiona.

\end{frame}


\begin{frame}
  \frametitle{Algorytm, krok 5.2: nachylenie prostego odcinka ST}
  
  Obliczenie nachylenia prostego odcinka ST jest dużo łatwiejsze, ponieważ
  wystarczy wykorzystać do tego nachylenie prostej łączącej punkty $J20$ i $TE$.

  Następnie wartość tego nachylenia jest poddawana ocenie. Jeśli jest większa od
  żądanego progu, odcinek ST wznosi się, jeżeli mniejsza od przeciwnej wartości
  progu: opada. Jeśli znajduje się pomiędzy tymi wartościami przyjmujemy, że
  odcinek jest poziomy.
\end{frame}

\begin{frame}
  \frametitle{Algorytm, krok 6: wykrywanie epizodów}


  Epizodami ST są zdarzenia polegające na utrzymaniu się parametrów ilościowych
  odcinka ST powyżej wartości progowych przez dłuższy czas.

  Według definicji Polskiego Towarzystwa Kardiologicznego epizod ST to odcinek
  czasu o długości $60s$ odległy od poprzedniego o $30s$ w którym uniesienie
  ciągle przekracza $1mm$ i odcinek narasta lub uniesienie przekracza $2mm$
  ($1mV = 10mm$).
\end{frame}

\begin{frame}
  \frametitle{Bibliografia}

  \begin{itemize}
    \item \emph{An Algorithm of ST Segment Classification and Detection}, Zhao
      Shen, Chao Hu, Jingsheng Liao, 2010, ICAL
    \item \emph{Przetwarzanie sygnałów elektrodiagnostycznych}, Piotr
      Augustyniak, 2001, AGH
  \end{itemize}
\end{frame}

\end{document}


